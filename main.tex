% Chapter titles correspond to new months.
% Section titles correspond to Journal entry dates.
% Subsection titles correspond to optional Journal entry titles.
% Note: journal entry files must escape "\" "{" and "}" if they are not used as a TeX operations. Other special
%       TeX symbols are automatically escaped.

\documentclass[12pt]{report}

% import preamble
\usepackage{journal}

% import index.tex which creates the variable \fileindex
% This index file is generated separately via an external script (and not natively through LaTeX).
% The script is "closejournal.sh" and is executed when I finish adding a new entry in the journal.
\newcommand*{\fileindex}{
content/2023/2023-06-10.tex,
content/2023/2023-06-09.tex,
content/2023/2023-06-08.tex,
content/2023/2023-06-07.tex,
content/2023/2023-06-04.tex,
content/2023/2023-05-23.tex,
content/2023/2023-05-21.tex,
content/2023/2023-05-20.tex,
content/2023/2023-05-19.tex,
content/2023/2023-04-27.tex,
content/2023/2023-04-26.tex,
content/2023/2023-04-24.tex,
content/2023/2023-04-20.tex,
content/2023/2023-03-29.tex,
content/2023/2023-03-12.tex,
content/2023/2023-03-07.tex,
content/2023/2023-03-02.tex,
content/2023/2023-02-12.tex,
content/2022/2022-06-09.tex,
content/2022/2022-06-08.tex,
content/2022/2022-06-07.tex,
content/2022/2022-06-04.tex,
content/2022/2022-05-07.tex,
content/2022/2022-05-04.tex,
content/2022/2022-04-20.tex,
content/2022/2022-04-08.tex
}


\begin{document}

% Title page: import from title.tex
% TITLE PAGE
\thispagestyle{empty} % Prevent header and footer from rendering
\begin{center}
	\vfill
    \vspace*{0.4\textheight}

	\Huge
	\bf{Journal}
    
	\Large
	\it{Firstname Lastname}

    \normalsize
\end{center}
\newpage

% Table of Contents
\tableofcontents
\thispagestyle{empty}

% for loop. "\file" is the variable. "\fileindex" is the source list.
\foreach \filepath in \fileindex {

	% Define date variables
	\StrMid{\filepath}{14}{17}[\ENTRYYEAR]
	\StrMid{\filepath}{19}{20}[\ENTRYMONTH]
	\StrMid{\filepath}{22}{23}[\ENTRYDAY]
	\def\DAYOFWEEK{\dayofweekname{\ENTRYDAY}{\ENTRYMONTH}{\ENTRYYEAR}}

	% Check if this is the first entry for the year:
	\IfStrEq{\ENTRYYEAR}{\currentyear}{
		% This is not the first entry for the year.
	}
	{
		% This is the first entry for the year, so create a new year page.
		% Add empty page with just the year on it. The newyear.tex contains a \phantomsection for anchoring links.
		\newpage
\thispagestyle{empty} % Prevent the header and footer from rendering
		\phantomsection
\begin{center}
	\vfill
	\vspace*{0.4\textheight}

	\Huge
	\bf{\ENTRYYEAR}

	\normalsize

%	\vspace{0.33\textheight}
\end{center}
\newpage

		% Set the header for the page. This corresponds with the fancyhead call in the preamble.
		\markboth{\ENTRYMONTH\hphantom{ }-\hphantom{ }\ENTRYYEAR}{}

		% Add new year to Table of Contents
		\addcontentsline{toc}{chapter}{\ENTRYYEAR}
		% Reset currentmonth back to zero. 
		% This fixes an example edge case: If my last journal entry was May 2020, and the very next journal entry is May 2021
		% then the check for whether this is the first entry of the month will not work properly.
		\global\let\currentmonth=0
	}

	% Check if this is the first entry for the month:
	\IfStrEq{\ENTRYMONTH}{\currentmonth}{}{

		\phantomsection
		% This is the first entry for the month, so add a month chapter. This line does not add it to the TOC.
		% Because of the "*", we need to tell LaTeX where the section anchor is, so we place a phantomsection just before this line.
		\chapter*{\monthname[\ENTRYMONTH],\hphantom{ }\ENTRYYEAR}

		% This is the first entry for the month. Add month to Table of Content
		\addcontentsline{toc}{section}{\ENTRYMONTH \hphantom{ } \monthname[\ENTRYMONTH]}

		% Set the global currentmonth variable to track which entry months we have published.
		\global\let\currentmonth=\ENTRYMONTH
	}

	% Journal entry date. Use of the "*" prevents adding the numbered section to the table of contents.
	\section*{\textit{\DAYOFWEEK, \ENTRYDAY\hphantom{ }\monthname[\ENTRYMONTH]\hphantom{ }\ENTRYYEAR}}

	% Set the header for the page. This corresponds with the fancyhead call in the preamble. The location of this is important.
	% If you modify this, you should also change the other call to \markboth. I kept the month numerical as it is easier to scroll
	% and find the desired date.
	\markboth{\ENTRYMONTH\hphantom{ }-\hphantom{ }\ENTRYYEAR}{Aim of this Thesis}

	% Print the journal entry content as raw input (escaping most TeX control sequences except \ { }.
	% We allow ``\ {  }'' to enable us to define an optional title.
	\escapedinput{\filepath}

	% Set the global currentyear variable to track which entry years we have published.
	\global\let\currentyear=\ENTRYYEAR
	% Set the global currentmonth variable to track which entry months we have published.
	\global\let\currentmonth=\ENTRYMONTH

}

\end{document}
