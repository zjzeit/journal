% Chapter titles correspond to new months.
% Section titles correspond to Journal entry dates.
% Subsection titles correspond to optional Journal entry titles.


% Preamble
\documentclass[12pt]{report}
\usepackage{pgffor} % Allows the foreach command.
\usepackage{xstring}
\usepackage{datetime} % For converting date names
\usepackage[tmargin=.75in,bmargin=1in]{geometry}
\usepackage{hyperref}  % For hyperlinks in the Table of Contents
\usepackage{lipsum}
\usepackage{expl3} % For regex matching
\usepackage{fancyhdr} % For custom header/footer
\usepackage{titlesec} % For format and spacing of chapter and section titles
\usepackage{indentfirst} % For indenting the first sentence in a paragraph. 

% Set the indent amount of the first sentence in a paragraph.
\setlength{\parindent}{25pt} 

% Set the format and spacing of chapter and section titles. This requires the titlesec package
\titleformat{\chapter}[display]
  {}{\Large\scshape\chaptertitlename\ \thechapter\\*[-15pt]\hrulefill}{-13pt}{\LARGE\bfseries}[\vspace{-25pt}]
\titleformat{\section}[display]
  {}{\Large\scshape\partname\ \thepart\\*[-15pt]\hrulefill}{-13pt}{\large\bfseries}[\vspace{0pt}]
\titleformat{\subsection}[display]
  {}{\Large\scshape\partname\ \thepart\\*[-15pt]\hrulefill}{-13pt}{\normalsize\bfseries}[\vspace{0pt}]
\titlespacing*{\chapter}
  {0pt}{10pt}{40pt}
\titlespacing*{\section}
  {0pt}{0pt}{5pt}
\titlespacing*{\subsection}
  {0pt}{0pt}{0pt}

% Create a reusable newread stream called filereadstream (there is a limit to the number of these).
% This is used when reading a Journal entry line-by-line.
\newread\filereadstream

% Enable hyperlinks for the Table of Contents.
\hypersetup{
    colorlinks,
    citecolor=black,
    filecolor=black,
    linkcolor=black,
    urlcolor=black
}

% Set global counters for the year and month that hav been printed. This is used to create year pages. It also helps for identifying first-of-month entries.
\global\let\currentyear=0
\global\let\currentmonth=0
\global\let\ENTRYMONTH=0
\global\let\ENTRYYEAR=0

% Header / Footer
\fancypagestyle{plain}{
  \fancyhf{}% Clear default header/footer
  \fancyhead[R]{\leftmark}% Right header. I update the "leftmark" manually using calls to "\markboth" in the document.
  \fancyfoot[R]{\thepage}% Right footer
}

% Set default page style to plain, which has associated headers and footers (as defined previously by \fancypagestyle).
\pagestyle{plain}

% import index.tex which creates the variable \fileindex
% This index file is generated separately via an external script (and not natively through LaTeX).
% The script is "closejournal.sh" and is executed when I finish adding a new entry in the journal.
\newcommand*{\fileindex}{
content/2023/2023-06-10.tex,
content/2023/2023-06-09.tex,
content/2023/2023-06-08.tex,
content/2023/2023-06-07.tex,
content/2023/2023-06-04.tex,
content/2023/2023-05-23.tex,
content/2023/2023-05-21.tex,
content/2023/2023-05-20.tex,
content/2023/2023-05-19.tex,
content/2023/2023-04-27.tex,
content/2023/2023-04-26.tex,
content/2023/2023-04-24.tex,
content/2023/2023-04-20.tex,
content/2023/2023-03-29.tex,
content/2023/2023-03-12.tex,
content/2023/2023-03-07.tex,
content/2023/2023-03-02.tex,
content/2023/2023-02-12.tex,
content/2022/2022-06-09.tex,
content/2022/2022-06-08.tex,
content/2022/2022-06-07.tex,
content/2022/2022-06-04.tex,
content/2022/2022-05-07.tex,
content/2022/2022-05-04.tex,
content/2022/2022-04-20.tex,
content/2022/2022-04-08.tex
}


\begin{document}

% Title page: import from title.tex
% TITLE PAGE
\thispagestyle{empty} % Prevent header and footer from rendering
\begin{center}
	\vfill
    \vspace*{0.4\textheight}

	\Huge
	\bf{Journal}
    
	\Large
	\it{Firstname Lastname}

    \normalsize
\end{center}
\newpage

% Table of Contents
\tableofcontents
\thispagestyle{empty}

% for loop. "\file" is the variable. "\fileindex" is the source list.
\foreach \filepath in \fileindex {

	% Define date variables
	\StrMid{\filepath}{14}{17}[\ENTRYYEAR]
	\StrMid{\filepath}{19}{20}[\ENTRYMONTH]
	\StrMid{\filepath}{22}{23}[\ENTRYDAY]
	\def\DAYOFWEEK{\dayofweekname{\ENTRYDAY}{\ENTRYMONTH}{\ENTRYYEAR}}

	% Check if this is the first entry for the year:
	\IfStrEq{\ENTRYYEAR}{\currentyear}{
		% This is not the first entry for the year.

		% Check if this is the first entry for the month:
		\IfStrEq{\ENTRYMONTH}{\currentmonth}{}{

			\phantomsection
			% This is the first entry for the month, so add a month chapter. This line does not add it to the TOC.
			% Because of the "*", we need to tell LaTeX where the section anchor is, so we place a phantomsection just before this line.
			\chapter*{\monthname[\ENTRYMONTH],\hphantom{ }\ENTRYYEAR}

			% This is the first entry for the month. Add month to Table of Content
			\addcontentsline{toc}{section}{\ENTRYMONTH \hphantom{ } \monthname[\ENTRYMONTH]}

			% Set the global currentmonth variable to track which entry months we have published.
			\global\let\currentmonth=\ENTRYMONTH
		}

		% Journal entry date
		\section*{\textit{\DAYOFWEEK, \ENTRYDAY\hphantom{ }\monthname[\ENTRYMONTH]\hphantom{ }\ENTRYYEAR}}

		% Set the header for the page. This corresponds with the fancyhead call in the preamble. The location of this is important.
		% If you modify this, you should also change the other call to \markboth. I kept the month numerical as it is easier to scroll through
		% and find the desired date.
		\markboth{\ENTRYMONTH\hphantom{ }-\hphantom{ }\ENTRYYEAR}{Aim of this Thesis}

		% Print the journal entry content
		% Use the newread stream (filereadstream is defined in the preamble). For leach line in the journal entry, check if "::title::" exists.
		\openin\filereadstream=\filepath
		\loop\unless\ifeof\filereadstream
			\read\filereadstream to \fileline % Reads a line of the file into \fileline
			\IfSubStr{\fileline}{::title::}{
				% The line in this journal entry contains "::title::" so parse it.
				\StrMid{\fileline}{11}{99}[\ENTRYTITLE]
				\phantomsection
				% Create the optional boldface title just above the journal entry content.
				\subsection*{\ENTRYTITLE}
				%  Add the title to the Table of Contents.
				\addcontentsline{toc}{subsection}{\ENTRYTITLE}
			}{
				% The current line in this journal entry does not contain "::title::" so simply print the line.
				% Check if the line is empty. If it is, print nothing. Else print the line.
				\setbox0=\hbox{\fileline\unskip}\ifdim\wd0=0pt
				% Pring nothing here, as the content from the file was an empty line. This prevents extra space at the end of each Journal entry.
				% It also prevents multiple spaces between paragraphs in the journal entry from rendering.
				{}
				\else
					% I add a newline (which requires some text before it, so I added an empty hphantom).
					\fileline \hphantom{}\newline
				\fi

			}
		\repeat
		\closein\filereadstream
	}{
		% This is the first entry for the year, so create a new year page.
		% Add empty page with just the year on it. The newyear.tex contains a \phantomsection for anchoring links.
		\newpage
\thispagestyle{empty} % Prevent the header and footer from rendering
		\phantomsection
\begin{center}
	\vfill
	\vspace*{0.4\textheight}

	\Huge
	\bf{\ENTRYYEAR}

	\normalsize

%	\vspace{0.33\textheight}
\end{center}
\newpage

		% Set the header for the page. This corresponds with the fancyhead call in the preamble.
		\markboth{\ENTRYMONTH\hphantom{ }-\hphantom{ }\ENTRYYEAR}{}

		% Add new year to Table of Contents
		\addcontentsline{toc}{chapter}{\ENTRYYEAR}

		% Because of the "*" in \chapter*, we need to tell LaTeX where the section anchor is, so we place a phantomsection.
		\phantomsection
		% This is the first entry for the month (automatically, because it's also the first entry for the year), so add a month chapter. 
		% This line does not add it to the TOC due to the "*". 
		\chapter*{\monthname[\ENTRYMONTH],\hphantom{ }\ENTRYYEAR}

		% This is the first entry for the month. Add month to Table of Content
		\addcontentsline{toc}{section}{\ENTRYMONTH \hphantom{ } \monthname[\ENTRYMONTH]}

		% Journal entry date. Use of the "*" prevents adding the numbered section to the table of contents.
		\section*{\textit{\DAYOFWEEK, \ENTRYDAY\hphantom{ }\monthname[\ENTRYMONTH]\hphantom{ }\ENTRYYEAR}}

		% Print the journal entry content
		% Use the newread stream (filereadstream is defined in the preamble). For leach line in the journal entry, check if "::title::" exists.
		\openin\filereadstream=\filepath
		\loop\unless\ifeof\filereadstream
			\read\filereadstream to \fileline % Reads a line of the file into \fileline
			\IfSubStr{\fileline}{::title::}{
				% The line in this journal entry contains "::title::" so parse it.
				\StrMid{\fileline}{11}{99}[\ENTRYTITLE]
				\phantomsection
				% Create the optional boldface title just above the journal entry content.
				\subsection*{\ENTRYTITLE}
				%  Add the title to the Table of Contents.
				\addcontentsline{toc}{subsection}{\ENTRYTITLE}
			}{
				% The current line in this journal entry does not contain "::title::" so simply print the line.
				% Check if the line is empty. If it is, print nothing. Else print the line.
				\setbox0=\hbox{\fileline\unskip}\ifdim\wd0=0pt
				% Pring nothing here, as the content from the file was an empty line. This prevents extra space at the end of each Journal entry.
				% It also prevents multiple spaces between paragraphs in the journal entry from rendering.
				{} 
				\else
					% I add a newline (which requires some text before it, so I added an empty hphantom).
					\fileline \hphantom{}\newline
				\fi

			}
		\repeat
		\closein\filereadstream

		% Set the global currentyear variable to track which entry years we have published.
		\global\let\currentyear=\ENTRYYEAR
		% Set the global currentmonth variable to track which entry months we have published.
		\global\let\currentmonth=\ENTRYMONTH
	}
}

\end{document}